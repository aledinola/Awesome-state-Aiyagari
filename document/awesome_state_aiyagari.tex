%\input{tcilatex}
%\input{tcilatex}
%\input{tcilatex}


\documentclass[12pt]{article}
%%%%%%%%%%%%%%%%%%%%%%%%%%%%%%%%%%%%%%%%%%%%%%%%%%%%%%%%%%%%%%%%%%%%%%%%%%%%%%%%%%%%%%%%%%%%%%%%%%%%%%%%%%%%%%%%%%%%%%%%%%%%%%%%%%%%%%%%%%%%%%%%%%%%%%%%%%%%%%%%%%%%%%%%%%%%%%%%%%%%%%%%%%%%%%%%%%%%%%%%%%%%%%%%%%%%%%%%%%%%%%%%%%%%%%%%%%%%%%%%%%%%%%%%%%%%
\usepackage[T1]{fontenc}
\usepackage[latin9]{inputenc}
\usepackage[a4paper]{geometry}
\usepackage{color}
\usepackage{array}
\usepackage{amsmath}
\usepackage{amssymb}
\usepackage{graphicx}
\usepackage[authoryear]{natbib}
\usepackage{ulem}
\usepackage{rotating}
\usepackage[unicode=true,
bookmarks=true,bookmarksnumbered=false,bookmarksopen=true,bookmarksopenlevel=1,
breaklinks=true,pdfborder={0 0 1},backref=section,colorlinks=true]{hyperref}
\usepackage{amsthm}
\usepackage{color}
\usepackage{amsfonts}
\usepackage[title]{appendix}

\setcounter{MaxMatrixCols}{10}
%TCIDATA{OutputFilter=Latex.dll}
%TCIDATA{Version=5.50.0.2953}
%TCIDATA{<META NAME="SaveForMode" CONTENT="1">}
%TCIDATA{BibliographyScheme=Manual}
%TCIDATA{LastRevised=Friday, August 01, 2025 18:12:03}
%TCIDATA{<META NAME="GraphicsSave" CONTENT="32">}

\geometry{verbose}
\PassOptionsToPackage{normalem}{ulem}
\hypersetup{
plainpages=false,urlcolor=magenta,citecolor=magenta,linkcolor=blue,pdfstartview=FitH,pdfview=FitH,plainpages=false,urlcolor=blue,citecolor=blue,linkcolor=blue,pdfstartview=FitH,pdfview=FitH}
\makeatletter
\graphicspath{{Figures/}} 
\setlength{\marginparwidth}{0in} \setlength{\marginparsep}{0in}
\setlength{\oddsidemargin}{0in} \setlength{\evensidemargin}{0in}
\setlength{\textwidth}{6.35in} \setlength{\topmargin}{-.50in}
\setlength{\textheight}{9.45in}
\renewcommand{\baselinestretch}{1.2}\small\normalsize
\providecommand{\tabularnewline}{\\}
\newcommand{\ssskip}{\vspace*{0.05cm}}
\newcommand{\sskip}{\vspace*{0.15cm}}
\newcommand{\lskip}{\vspace*{0.45cm}}
\makeatother
\input{tcilatex}
\begin{document}

\title{\textbf{Awesome-state Aiyagari}}
\date{\today }
\author{}
\maketitle

\begin{itemize}
\item Notation. Here $\Pi ^{e}$ denotes the transition matrix of the
idiosyncratic shock $e$, $G^{e}$ denotes the \textit{initial} distribution
of $e$. Pension benefits are denoted as $\omega _{ret}$.

\item Young individual's problem:%
\begin{equation*}
V^{Y}(a,e)=\max_{l,a^{\prime }}\left\{ \frac{\left( c^{\gamma }\left( 1-\ell
\right) ^{1-\gamma }\right) ^{1-\sigma }}{1-\sigma }+\beta \left(
1-p_{ret}\right) \sum_{e^{\prime }}V^{Y}(a^{\prime },e^{\prime })\Pi
^{e}(e,e^{\prime })+\beta p_{ret}V^{R}(a^{\prime })\right\} 
\end{equation*}%
subject to%
\begin{equation*}
c+a^{\prime }=we\ell +(1+r)a,
\end{equation*}%
\begin{equation*}
a^{\prime }\geq 0.
\end{equation*}%
Policy functions: $a^{\prime }=g_{a^{\prime }}^{Y}(a,e)$ and $\ell =g_{\ell
}^{Y}(a,e)$.

\item Old individual's problem:%
\begin{equation*}
V^{R}(a)=\max_{a^{\prime }}\left\{ \frac{c^{1-\sigma }}{1-\sigma }+\beta
p_{death}\sum_{e^{\prime }}V^{Y}(a^{\prime },e^{\prime })G^{e}(e^{\prime
})+\beta \left( 1-p_{death}\right) V^{R}(a^{\prime })\right\} 
\end{equation*}%
subject to%
\begin{equation*}
c+a^{\prime }=\omega _{ret}+(1+r)a,
\end{equation*}%
\begin{equation*}
a^{\prime }\geq 0.
\end{equation*}%
Policy function: $a^{\prime }=g_{a^{\prime }}^{R}(a)$ (and implicitly $\ell
=g_{\ell }^{R}(a,e)=0$).

\item Stationary distribution $\left[ \mu ^{Y}(a,e),\mu ^{R}(a)\right] $
satisfies%
\begin{equation*}
\mu ^{Y}(a^{\prime },e^{\prime })=\left( 1-p_{ret}\right) \sum_{a}\sum_{e}%
\mathbb{I}_{\left\{ a^{\prime }=g_{a^{\prime }}^{Y}\left( a,e\right)
\right\} }\Pi ^{e}(e,e^{\prime })\mu ^{Y}(a,e)+p_{death}\sum_{a}\mathbb{I}%
_{\left\{ a^{\prime }=g_{a^{\prime }}^{R}\left( a\right) \right\}
}G^{e}(e^{\prime })\mu ^{R}(a)
\end{equation*}%
\begin{equation*}
\mu ^{R}(a^{\prime })=p_{ret}\sum_{a}\sum_{e}\mathbb{I}_{\left\{ a^{\prime
}=g_{a^{\prime }}^{Y}\left( a,e\right) \right\} }\mu
^{Y}(a,e)+(1-p_{death})\sum_{a}\mathbb{I}_{\left\{ a^{\prime }=g_{a^{\prime
}}^{R}\left( a\right) \right\} }\mu ^{R}(a)
\end{equation*}

\item \textbf{Aggregate variables}. Firms operate a Cobb-Douglas production
function:%
\begin{equation*}
Y=K^{\alpha }L^{1-\alpha }
\end{equation*}%
Prices are determined competitively:%
\begin{equation*}
r=\alpha \left( \frac{K}{L}\right) ^{\alpha -1}-\delta \text{, and }w=\left(
1-\alpha \right) \left( \frac{K}{L}\right) ^{\alpha }
\end{equation*}%
Aggregate capital is%
\begin{equation*}
K=\sum_{a}\sum_{e}ad\mu \left( a,e\right) 
\end{equation*}%
Aggregate labor is 
\begin{equation*}
L=\sum_{a}\sum_{e}e\times g_{\ell }^{Y}(a,e)d\mu \left( a,e\right) 
\end{equation*}

\item \textbf{General equilibrium loop}. We follow this iterative algorithm:

\begin{enumerate}
\item Guess interest rate $r$

\item Compute the wage implied by the factor pricing conditions:%
\begin{equation*}
w=\left( 1-\alpha \right) \left( \frac{\alpha }{r+\delta }\right) ^{\frac{%
\alpha }{1-\alpha }}
\end{equation*}

\item Given $\left\{ r,w,\Pi ^{e}(e,e^{\prime }),G^{e}(e^{\prime
}),p_{ret},p_{death}\right\} $, compute policy functions $g_{a^{\prime
}}^{Y}(a,e)$, $g_{\ell }^{Y}(a,e)$ and $g_{a^{\prime }}^{R}(a)$. 

\item Given $\left\{ g_{a^{\prime }}^{Y}(a,e),g_{a^{\prime }}^{R}(a),\Pi
^{e}(e,e^{\prime }),G^{e}(e^{\prime }),p_{ret},p_{death}\right\} $, compute
invariant distribution $\mu \left( a,e\right) $.

\item Compute implied interest rate%
\begin{eqnarray*}
\widehat{K} &=&\sum_{a}\sum_{e}ad\mu \left( a,e\right)  \\
\widehat{L} &=&\sum_{a}\sum_{e}e\times g_{\ell }^{Y}(a,e)d\mu \left(
a,e\right) 
\end{eqnarray*}
\begin{equation*}
\widehat{r}=\alpha \left( \frac{\widehat{K}}{\widehat{L}}\right) ^{\alpha
-1}-\delta 
\end{equation*}

\item If $\left\vert r-\widehat{r}\right\vert <tol$, then stop. Otherwise,
update the interest rate $r$ and go back to step 2.
\end{enumerate}

\item \textbf{Calibration}. Guvenen et al. (2023) assume that newborn
individuals draw their initial labor efficiency from the initial
distribution $G^{e}$. Guvenen et al. (2023) set $n_{e}=4$, and take the
values $e_{1}$,$e_{2},e_{3}$ and the $4\times 4$ transition matrix $\Pi ^{e}$
from Castaneda, Diaz-Gimenez and Rios-Rull (2003, Tables 4-5). They
calibrate $e_{n_{e}}$, the value of the \textquotedblleft
awesome\textquotedblright\ state, to match a share of wealth held by the top
1\% equal to $30\%$. They calibrate $\omega _{ret}$ so that the ratio of
pension benefits to GDP is equal to $4.9\%$. For simplicity, they set $G^{e}$
equal to the stationary distribution of labor efficiency, $\gamma ^{e}$.

\item \textbf{VFI toolkit}. We combine the two exogenous shocks $e=\left[
e_{1},\ldots ,e_{4}\right] $ and $age=[Y,R]$ into a single array%
\begin{equation*}
z\_grid=%
\begin{bmatrix}
e_{1} & Y \\ 
e_{2} & Y \\ 
e_{3} & Y \\ 
e_{4} & Y \\ 
0 & R%
\end{bmatrix}%
\end{equation*}
with size $[n_{e}+1,2]$. We set the transition matrix for this newly defined
exogenous shock $z$ as follows:%
\begin{equation*}
\underbrace{\Pi ^{z}}_{2n_{e}\times 2n_{e}}=%
\begin{bmatrix}
\left( 1-p_{ret}\right) \underbrace{\Pi ^{e}(e,e^{\prime })}_{n_{e}\times
n_{e}} & p_{ret}I_{n_{e}} \\ 
p_{death}\underbrace{ones(n_{e},1)G^{e}}_{n_{e}\times n_{e}} & \left(
1-p_{death}\right) I_{n_{e}}%
\end{bmatrix}%
\end{equation*}%
where $I_{n_{e}}$ denotes the $n_{e}\times n_{e}$ identity matrix and $%
ones(n_{e},1)G^{e}$ denotes a matrix with $G^{e}$ in every row. Let $%
n_{z}=2n_{e}$. The toolkit will compute the arrays $V$, $Policy$ and $%
StationaryDist$%
\begin{equation*}
V:n_{a}\times n_{e}\times n_{age}\text{ array}
\end{equation*}%
\begin{equation*}
Policy:2\times n_{a}\times n_{e}\times n_{age}\text{ array}
\end{equation*}%
\begin{equation*}
StationaryDist:n_{a}\times n_{e}\times n_{age}\text{ array}
\end{equation*}%
Note that 
\begin{eqnarray*}
g_{l}(a,e,age) &=&Policy(1,a,e,age), \\
g_{a^{\prime }}(a,e,age) &=&Policy(2,a,e,age),
\end{eqnarray*}%
where $a=1,..,n_{a}$, $e=1,..,n_{e}$ and $age=Y,R$. Note that e.g. $V(a,e,R)$
must be the same for all $e$, since once the individual is retired the only
relevant state variable is $a$, not labor productivity.
\end{itemize}

\end{document}
